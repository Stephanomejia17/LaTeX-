\documentclass[•]{article}
\begin{document}
	\begin{huge}
		\begin{center}
			\textbf{Vol\'umenes}
		\end{center}
	\end{huge}
	
	
	\begin{flushleft}
	
		\textbf{Definiciones: }
		\begin{itemize}
			\item \textbf{\'Angulo Diedro: } El \'angulo diedro es la figura geom\'etrica que forman dos semiplanos determinados por una misma recta y dicha recta.
			
			\item \textbf{Medida del \'angulo diedro: } Para hallar la medida de un \'angulo diedro debe trazarse desde un punto cualquiera de la arista, dos perpendiculares levantadas, cada una contenida en una de las caras del \'angulo, dichas perpendiculares son rayos con el mismo origen, es decir que forman un \'angulo. La medida del \'angulo diedro es la misma medida del \'angulo que forman dichos rayos.
			
			\item Dos \'angulos diedros son congruentes si sus medidas son iguales.
			
			\item \textbf{Poliedro: } Un poliedro es una figura tridimensional cerrada (s\'olido), formada por la uni\'on de refiones poligonales que se intersecan \'unicamente en sus lados. Las regiones poligonales se denominan caras, las intersecciones de dos de las caras, aristas y las intersecciones de m\'as de dos caras, v\'ertices. Puede decirse entonces que un poliedro es un s\'olido de caras planas.
			
			\item \textbf{Poliedro regular: }Un poliedro regular es aquel cuyas caras son pol\'igonos regulares congruentes y los \'angulos diedros tambi\'en son congruentes. 
			\item \textbf{Tetraedro regular: }poliedro regular de cuatro caras (tri\'angulos equil\'ateros), seis aristas y cuatro v\'ertices que generan \'angulos triedros (de tres caras). Representa el fuego.
			
			\item \textbf{Hexaedro regular o cubo: }poliedro regular de seis caras cuadradas, doce aristas y ocho v\'ertices que generan \'angulos triedros. Representa el aire.
			\item \textbf{Dodecaedro regular: }poliedro regular de doce caras pentagonales, 30 aristas y 20 v\'ertices que generan \'angulos triedros. Representa el cielo.
			
			\item \textbf{Icosaedro regular: }Poliedro regular de 20 caras triangulares, 30 aristas y doce v\'ertices que generan \'angulos pentaedros. Representa el agua.
			
			\item \textbf{Prisma: }Un prisma es un poliedro con dos caras congruentes y paralelas llamadas bases y cuyas otras caras laterales son paralelogramos. La altura del prisma es el segmento perpendicular a la bases y cuyos extremos pertenecen a las bases o a los planos que las contienen.
			
			\item \textbf{Pi\'amide: }Una pir\'amide es un poliedro que tiene una cara como base, y el resto de sus caras son tri\'angulos que comparten un v\'ertice llamado v\'ertice de la pir\'amide. La altura de la pir\'amide es el segmento perpendicular a la base (o al plano que la contiene que tiene un extremo en el v\'ertice de la pir\'amide y el otro pertenece a la base (o al plano que la contiene).
			
			\item El \'area superficial de un poliedro (o \'area de la superficie) es igual a la suma de las \'areas de las regiones poligonales que conforman sus caras.
			
			\item El \'area superficial de un prisma es la suma de las \'areas de sus caras laterales m\'as el doble producto del \'area de su base.
			
			\item \textbf{Esfera: }Una esfera es el conjunto de todos los puntos en el espacio que se obtienen al hacer girar un cemic\'irculo alrededor de su di\'ametro.
			
			\item \textbf{Cilindro: }Un cilindro es el s\'olido formado por todos los puntos del espacio limitados por dos c\'irculos paralelos y congruentes y la figura obtenida al hacer girar un segmento que va de circunferencia a circunferencia, alrededor de esta. El segmento que une los centros de los c\'irculos se llama eje y el segmento que gira es la generatriz.
			\item \textbf{Cilindro recto: }Un cilindro recto es aquel en el que la generatriz es perpendicular a las bases.
			\item La altura de un cilindro es el segmento perpenducular a las bases (y su medida) y cuyos puntos extremos pertenecen a los planos qjue contienen a las bases. 
			
			\item \textbf{Cono: }Un cono es el s\'olido limitado por un c\'irculo y todos los segmentos que parten de cada uno de los puntos de la circunferencia y van a un \'unico punto, no coplanar con el c\'irculo, llamdo v\'ertice.
			
			\item \textbf{Cono recto: }Un cono recto es aquel en el que el segmento que va del v\'ertice al centro del c\'irculo, es perpendicular a \'este.
			
			\item La altura de un cono es el segmento perpendicular a la base (y su medida), que parte del v\'ertice y su otro punto extremo pertenece al plano que contiene a la base.
		\end{itemize}				
		
		
		
		\textbf{Teoremas: }
		
		\begin{itemize}
			\item El volumen de una pir\'amide inscrita en un prisma es igual a la tercera parte del volumen del prisma.
			
			\item El volumen de un cono inscrito en un cilindro es igual a la tercera parte del volumen de dicho cilindro. $V_{cono}=\frac{1}{3}(V_{cilindro})$ \\
			$V_{cono}=\frac{1}{3}(h\pi r^2)$
			\item El volumen de una esfera es igual al producto de los cuatro tercios de $\pi$ por el cubo del radio. $V_{Esfera} = \frac{4}{3}(\pi r^3)$ 
		\end{itemize}
		
		\textbf{Postulados: }
		\begin{itemize}
			\item A todo so\'lido se le asocia un n\'umero real no negativo llamado volumen.
			
			\item Un cubo cuyas aristas miden uno tiene un volumen igual a una unidad c\'ubica.
			
			\item \textbf{Adici\'on de vol\'umenes: } Si la intersecci\'on de dos s\'olidos es una regi\'on poligonal, un segmento, un punto o vac\'ia entonces el volumen de la uni\'on de los s\'olidos es igual a la suma de los vol\'umenes de dichos s\'olidos.
			
			
		\end{itemize}
	\end{flushleft}



\end{document}