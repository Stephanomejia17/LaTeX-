\documentclass[•]{article}

\begin{document}
	\begin{center}
		\Large{\textbf{Tri\'angulos y Desigualdades}}
	\end{center}

\textbf{Teoremas:}	
\begin{itemize}
	
	\item $\textbf{\'Angulo Exterior}$: La medida de un \'angulo exterior de un tri\'angulo es mayor que la medida de cualesquiera de los \'angulos interiores no adyacentes.
	\item $\textbf{Lado, \'angulo, \'angulo (L-A-A)}$: Si dos tri\'angulos $\bigtriangleup ABC$ y $\bigtriangleup DEF$ tienen una correspondencia tal que: $\overline{AB} \cong  \overline{DE}; \hat{A} \cong \hat{D}; \hat{C} \cong \hat{F}$; entonces los tri\'angulos son congruentes.
	\item En todo tri\'angulo is\'osceles la altura relativa a la base es tambi\'en mediana y est\'a contenida en la mediatriz de la base y en la bisectriz del \'angulo opuesto.
	\item Si dos lados de un tri\'angulo no son congruentes, los \'angulos opuestos tampoco lo son y a mayor lado se opone mayor \'angulo.
	\item Si dos \'angulos de un tri\'angulo no son congruentes, los lados opuestos tampoco lo son, y a mayor \'angulo se opone mayor lado.
	\item $\textbf{Desigualdad triangular}$: La suma de las medidas  de dos lados cualesquiera de un tri\'angulo es mayor que la medida del tercer lado.
	\item $\textbf{Lado - Lado - \'Angulo (L-L-A)}$: Si dos tri\'angulos $\bigtriangleup ABC$ y $\bigtriangleup DEF$ tiene una correspondencia tal que: $\overline{AB} \cong \overline{DE}$; $\overline{BC} \cong \overline{EF}$; $\hat{C} \cong \hat{F}$ y $AB > BC$; entonces los tri\'angulos son congruentes.
	\item $\textbf{Teorema de la Bisagra}$: Si dos tri\'angulos $\bigtriangleup ABC$ y $\bigtriangleup DEF$ tienen dos lados respectivamente congruentes y los \'angulos comprendidos por ellos desiguales, entonces a mayor \'angulo se opone mayor lado.
	\item Si dos tri\'angulos $\bigtriangleup ABC$ y $\bigtriangleup DEF$ tienen dos lados respectivamente congruentes y los terceros lados desiguales, entonces a mayor lado se opone mayor \'angulo

\end{itemize}

\textbf{Corolarios:}
\begin{itemize}
	\item Si un tri\'angulo tiene un \'angulo recto entonces los otros dos son agudos.
	\item $(H-A)$ Dos tri\'angulos rect\'angulos que tienen respectivamente congruentes la hipotenusa y un \'angulo agudo son congruentes.
	\item El segmento m\'as corto que une un punto a una recta es el segmento que va desde el punto a la recta, perpendicularmente
	\item La medida de la hipotenusa de un tri\'angulo rect\'angulo es mayor que la medida de cualesquiera de sus catetos
	\item $(H-C)$ Dos tri\'angulos rect\'angulos que tienen respectivamente congruentes la hipotenusa y un cateto son congruentes

\end{itemize}




\end{document}