\documentclass[•]{article}

\begin{document}
	\begin{center}
		\Large{\textbf{Tri\'angulos y Desigualdades}}
	\end{center}

\textbf{Teoremas:}	
\begin{itemize}
	\item Dos tri\'angulos rect\'angulos cuyos catetos correspondientes son respectivamente congruentes, son congruentes.
	\item \textbf{Pons Asinorum:} En todo tri\'angulo is\'osceles los \'angulos de la base son congruentes.
	\item En todo tri\'angulo is\'osceles la mediana a la base es a su vez altura y est\'a contenida en la mediatriz y en la bisectriz del \'angulo opuesto.
	\item Todo punto de la mediatriz de un segmento equidista de los extremos de \'este
	\item \textbf{\'Angulo - Lado - \'Angulo (A-L-A):} Dos tri\'angulos que tienen dos \'angulos correspondientes respectivamente congruentes y el lado compartido por dichos \'angulos respectivamente congruentes, son congruentes. 
	\item \textbf{Rec\'iproco de Pons Asinorum:} Si dos \'angulos de un tri\'angulo son congruentes, el tri\'angulo es is\'osceles
	\item En todo tri\'angulo is\'osceles la bisectriz del \'angulo opuesto a la base es tambi\'en mediana y altura, y est\'a contenida en la mediatriz de la base.
	\item \textbf{Lado - Lado - Lado (L-L-L):} Dos tri\'angulos son congruentes si tienen sus tres lados correspondientes respectivamente congruentes.
	\item $\textbf{\'Angulo Exterior}$: La medida de un \'angulo exterior de un tri\'angulo es mayor que la medida de cualesquiera de los \'angulos interiores no adyacentes.
	\item $\textbf{Lado, \'angulo, \'angulo (L-A-A)}$: Si dos tri\'angulos $\bigtriangleup ABC$ y $\bigtriangleup DEF$ tienen una correspondencia tal que: $\overline{AB} \cong  \overline{DE}; \hat{A} \cong \hat{D}; \hat{C} \cong \hat{F}$; entonces los tri\'angulos son congruentes.
	\item En todo tri\'angulo is\'osceles la altura relativa a la base es tambi\'en mediana y est\'a contenida en la mediatriz de la base y en la bisectriz del \'angulo opuesto.
	\item Si dos lados de un tri\'angulo no son congruentes, los \'angulos opuestos tampoco lo son y a mayor lado se opone mayor \'angulo.
	\item Si dos \'angulos de un tri\'angulo no son congruentes, los lados opuestos tampoco lo son, y a mayor \'angulo se opone mayor lado.
	\item $\textbf{Desigualdad triangular}$: La suma de las medidas  de dos lados cualesquiera de un tri\'angulo es mayor que la medida del tercer lado.
	\item $\textbf{Lado - Lado - \'Angulo (L-L-A)}$: Si dos tri\'angulos $\bigtriangleup ABC$ y $\bigtriangleup DEF$ tiene una correspondencia tal que: $\overline{AB} \cong \overline{DE}$; $\overline{BC} \cong \overline{EF}$; $\hat{C} \cong \hat{F}$ y $AB > BC$; entonces los tri\'angulos son congruentes.
	\item $\textbf{Teorema de la Bisagra}$: Si dos tri\'angulos $\bigtriangleup ABC$ y $\bigtriangleup DEF$ tienen dos lados respectivamente congruentes y los \'angulos comprendidos por ellos desiguales, entonces a mayor \'angulo se opone mayor lado.
	\item Si dos tri\'angulos $\bigtriangleup ABC$ y $\bigtriangleup DEF$ tienen dos lados respectivamente congruentes y los terceros lados desiguales, entonces a mayor lado se opone mayor \'angulo.
	
	\item \textbf{Paralela media:} El segmento que une los puntos medios de dos lados de un tri\'angulo es paralelo al tercer lado y mide la mitad de dicho lado
	\item \textbf{Punto medio - paralela: }Una recta que biseca a uno de los lados de un tri\'angulo y es paralela al otro lado del tri\'angulo, biseca tambi\'en al tercer lado.
	\item La suma de las medidas de los \'angulos interiores de un tri\'angulo es 180.
	\item La medida de cualquier \'angulo exterior de un tri\'angulo es igual a la suma de las medidas de los \'angulos interiores no adyacentes.
	\item \textbf{Teorema 30-60-90: }En un tri\'angulo rect\'angulo de medida 30, el cateto opuesto a dicho \'angulo mide la mitad de lo que mide la hipotenusa.
	\item \textbf{Teorema de la mediana relativa: }En todo tri\'angulo rect\'angulo, la mediana relativa a la hipotenusa mide la mitad de la medida de la hipotenusa.

\end{itemize}

\textbf{Postulados:}
\begin{itemize}
	\item Lado-\'Angulo-Lado(L-A-L): Dos tri\'angulos son congruentes si tienen respectivamente congruentes dos lados y el \'angulo comprendido entre ellos

\end{itemize}

\textbf{Corolarios:}
\begin{itemize}
	\item Un tri\'angulo es equil\'atero si es equi\'angulo.
	\item (C-A agudo): Dos tri\'angulos rect\'angulos que tienen respectivamente congruentes un cateto y un \'angulo agudo son congruentes.
	\item Si un tri\'angulo tiene un \'angulo recto entonces los otros dos son agudos.
	\item $(H-A)$ Dos tri\'angulos rect\'angulos que tienen respectivamente congruentes la hipotenusa y un \'angulo agudo son congruentes.
	\item El segmento m\'as corto que une un punto a una recta es el segmento que va desde el punto a la recta, perpendicularmente
	\item La medida de la hipotenusa de un tri\'angulo rect\'angulo es mayor que la medida de cualesquiera de sus catetos
	\item $(H-C)$ Dos tri\'angulos rect\'angulos que tienen respectivamente congruentes la hipotenusa y un cateto son congruentes.
	\item En un tri\'angulo rect\'angulo los \'angulos agudos son complementarios.
	\item Si dos tri\'angulos tienen dos \'angulos respectivamente congruentes, los terceros \'angulos son tambi\'en congruentes.

\end{itemize}




\end{document}