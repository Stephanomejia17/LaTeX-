\documentclass[•]{article}

\begin{document}
	\title{Accidentes Gramaticales del Verbo}
	
	\begin{center}
	\textbf{\Large{Accidentes Gramaticales del Verbo}}
	\end{center}
	

\begin{flushleft}\

Introducci\'on:
El verbo esta formado por un lexema o ra\'iz (que es la parte que no cambia del verbo) y por morfemas o terminaciones que son los que nos indican los cambios o accidentes gramaticales
	
	\textbf{Ejemplos:}
	\begin{itemize}
	\item estudia-as
	
	"estudi" ser\'ia nuestro lexema y "as" ser\'ia nuestro morfema
	\item vivi-mos
	\end{itemize}
	
	\textbf{Accidentes Gramaticales: }
	\begin{itemize}
	
	%PERSONA
	\item Persona
	\begin{itemize}
		\item PRIMERA PERSONA
		
		Yo entreno todos los d\'ias
		\item SEGUNDA PERSONA
		
		Camilo, juegas muy bien al tenis
		\item TERCERA PERSONA
		
		Todos los d\'ias mi vecino practica c\'omo cantar
	\end{itemize}
	%NUMERO
	\item N\'umero
	
	Cuando la accion que expresa el verbo es realizada por un sujeto (\textbf{SINGULAR}) o cuando es realizada por varias personas (\textbf{PLURAL})
	
	\textbf{SINGULAR:}
	Me gusta ir de paseo por el mall \\
	\textbf{PLURAL: }
	Debemos caminar diez cuadras para llegar a la Universidad de Medellin :)
	
	%TIEMPO
	\item Tiempo
	
	\begin{itemize}
		\item Tiempo Pasado: Expresa una acci\'on ya realizada \\ Juan comi\'o
		
		\item Tiempo Presente: La acci\'on se est\'a realizando actualmente \\ Juan corre
		\item Tiempo Futuro: Lo expresado por el verbo a\'un no ocurre, sino que ocurrir\'a posteriormente \\ Julia bailar\'a en el teatro
	\end{itemize}
	%MODO
	\item Modo
	\begin{itemize}
		\item Modo indicativo: El verbo se refiere a una acci\'on concreta y real
		
		Ejm: Anal\'ia cant\'o muy bien, nos informa de un hecho real y particular
		
		\item Modo Subjuntivo: Cuando el verbo se refiere a una acci\'on que se considera como posible, a\'un no concretada \\
		Ejm: Ojal\'a podamos asistir a su graduaci\'on
		
		\item Modo Imperativo: Cuando queremos dirigir \'ordenes \\ Ejm: Ven aqu\'i		
		
		
	\end{itemize}		
	
	
	%VOZ
	\item Voz
	
	Nos indica si el sujeto ejecuta (sujeto agente) o recibe (sujeto paciente) la accion del verbo
	\begin{itemize}
		\item Voz Activa: Cuando el sujeto ejecuta la acci\'on \\ Ejm: El estudiante contesta las preguntas
		\item Voz Pasiva: Cuando el sujeto recibe la acci\'on del verbo
	\end{itemize}		
	
	
	\end{itemize}
	
	
\end{flushleft}


\end{document}