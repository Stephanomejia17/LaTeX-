\documentclass[•]{article}
\begin{document}
	\begin{center}
		\begin{huge}
			\textbf{\'Areas de Figuras Planas}
		\end{huge}
	\end{center}
	
	\begin{flushleft}
		\textbf{Teoremas:}
		\begin{itemize}
			\item El \'area de un cuadrado es igual al cuadrado de la medida de uno de sus lados. 
			\item El \'area de un paralelogramo es el producto de la medida de un lado por la altura correspondiente (base por altura).
			\item El \'area de un tri\'angulo es el semi-producto de la longitus de cualquier base por su correspondiente altura.
		\end{itemize}
		
		
		\textbf{Postulados: }
		\begin{itemize}
			\item \textbf{Postulado de \'area:} Dada la unidad del \'area, a cada regi\'on le corresponde un \'unico n\'umero, llamado el \'area de la regi\'on.
			\item \textbf{De congruencia: }Si dos figuras geom\'etricas son congruentes, generan regiones que tiene la misma \'area.
			\item \textbf{Adici\'on de \'areas: }Si laintersecci\'on de dos regiones poligonales es un segmento, un punto o vac\'ia, entonces el \'area de su uni\'on es la suma de sus \'areas.
			\item \textbf{De la unidad: }El \'area de una regi\'on rectangular es el producto de su base por la altura.
		\end{itemize}
		
		
		
		\textbf{Corolarios: }
		\begin{itemize}
			\item El \'area de un tri\'angulo es el semiproducto de las longitudes de los catetos. 
		\end{itemize}
	\end{flushleft}


\end{document}