\documentclass[•]{article}
\begin{document}
	\begin{center}
		\begin{huge}
			\textbf{\'Areas de Figuras Planas}
		\end{huge}
	\end{center}
	
	\begin{flushleft}
		\textbf{Teoremas:}
		\begin{itemize}
			\item El \'area de un cuadrado es igual al cuadrado de la medida de uno de sus lados. 
			\item El \'area de un paralelogramo es el producto de la medida de un lado por la altura correspondiente (base por altura).
			\item El \'area de un tri\'angulo es el semi-producto de la longitus de cualquier base por su correspondiente altura.
			\item El \'area de un trapecio es el producto de la semi-suma de las medidas de las bases por la medida de la altura.
			\item El \'area de un rombo es el semi-producto de la medida de sus diagonales.
			\item \textbf{F\'ormula de Heron: }Si $a, b, c$ son las medidas de los lados de un tri\'angulo $ABC$ y $S$ es su semi-per\'imetro, entonces $A_{\triangle ABC} = \sqrt{S(S - a)(S - b)(S - c)}$. Recuerde que $S = \frac{perimetro}{2}$
			
			\item El \'area de pol\'igono regular es igual al semi-producto del n\'umero de lados del pol\'igono $(n)$ por la longitud del lado $(\ell)$ por el apotema $(a)$. $A_{poligono} = \frac{n . \ell . a}{2} $
			
			\item El \'area de un c\'irculo (regi\'on circular) es igual al producto de $\pi$ por el radio al cuadrado. $A_{C(O, r)} = \pi . r^2$
			
			\item El \'area de un sector circular $AOB$ de \'angulo central $\alpha$ est\'a dada por el producto de la porci\'on que representa dicho sector por el \'area del c\'irculo. $A_{sector.circ.AOB} = \frac{\alpha}{360} . \pi . r^2$
			
			\item Si dos tri\'angulos tienen la misma altura (o base), entonces la raz\'on entre sus \'areas es igual a la raz\'on entre sus bases (o alturas).
			
			\item Si dos tri\'angulos son semejantes, entonces la raz\'on entre sus \'areas es igual al cuadrado de la raz\'on de dos cualesquiera de sus elementos correspondientes.
		\end{itemize}
		
		
		\textbf{Postulados: }
		\begin{itemize}
			\item \textbf{Postulado de \'area:} Dada la unidad del \'area, a cada regi\'on le corresponde un \'unico n\'umero, llamado el \'area de la regi\'on.
			\item \textbf{De congruencia: }Si dos figuras geom\'etricas son congruentes, generan regiones que tiene la misma \'area.
			\item \textbf{Adici\'on de \'areas: }Si laintersecci\'on de dos regiones poligonales es un segmento, un punto o vac\'ia, entonces el \'area de su uni\'on es la suma de sus \'areas.
			\item \textbf{De la unidad: }El \'area de una regi\'on rectangular es el producto de su base por la altura.
		\end{itemize}
		
		
		
		\textbf{Corolarios: }
		\begin{itemize}
			\item El \'area de un tri\'angulo es el semiproducto de las longitudes de los catetos. 
		\end{itemize}
	\end{flushleft}


\end{document}