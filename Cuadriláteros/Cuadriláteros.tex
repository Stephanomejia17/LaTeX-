\documentclass[•]{article}
\begin{document}
\begin{center}
\large{Cuadriláteros}
\end{center}

\begin{flushleft}
	\textbf{Teoremas:}
	\begin{itemize}
		\item La suma de los \'angulos internos de un pol\'igono de n lados con interior convexo es $180 . (n-2)$.
		\item Los \'angulos de la base mayor de un trapecio is\'osceles son congruentes.
		\item \textbf{Teorema de la base media: } La base de un trapecio es paralela a las otras bases y su medida es la semisuma de las medidas de estas.
		\item Una diagonal de un paralelogramo determina dos tri\'angulos congruentes.
		\item Un cuadril\'atero es un paralelogramo si y s\'olo si los lados opuestos son congruentes.
		\item Un cuadril\'atero es un paralelogramo si y s\'olo si los \'angulos opuestos son congruentes.
		\item Dos \'angulos consecutivos cualesquiera en un paralelogramo son suplementarios.
		\item Un cuadril\'atero es un paralelogramo si y s\'olo si tiene un par de lados opuestos paralelos y congruentes.
		\item Un cuadril\'atero es un paralelogramo si y s\'olo si las diagonales se bisecan.
	\end{itemize}



\end{flushleft}





\end{document}