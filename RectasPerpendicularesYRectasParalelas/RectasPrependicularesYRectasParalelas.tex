\documentclass[•]{article}

\begin{document}

\begin{center}
\Large{Rectas Perpendiculares y Rectas Paralelas}
\end{center}


\textbf{Teoremas: \\ \\ Rectas Perpendiculares:}

\begin{itemize}
	\item \textbf{Perpendicular levantada: } Dada una recta sobre un plano $\pi$, por un punto cualquiera de la recta pasa una y solo una recta en $\pi$, perpendicular a ella.
	\item \textbf{Perpendicular bajada: } Por un punto exterior a una recta pasa una \'unica perpendicular a dicha recta.
	\item Si dos rectas en un plano son perpendiculares a la misma recta, ellas son paralelas entre s\'i.
	

\end{itemize}

\textbf{Rectas Paralelas:}
\begin{itemize}
\item \textbf{Postulado de Euclides: }La paralela que pasa por un punto exterior a una recta es \'unica
\item \textbf{Distancia entre paralelas: } Dadas dos paralelas, la distancia entre ellas es constante.

\item \textbf{TEOREMA FUNDAMENTAL DE PARALELISMO (T.F.P)} Si tres o m\'as rectas paralelas determinan segmentos congruentes en una transversal, entonces determinan segmentos congruentes sobre cualquier transversal
\end{itemize}

\textbf{\'Angulos:}
\begin{itemize}
\item Los \'angulos alternos internos formados por dos rectas $\ell_1$ y $\ell_2$ y una transversal $t$ son congruentes si y solo si $\ell_1$ y $\ell_2$ son paralelas ($\ell_1  ||  \ell_2$).

\item Los \'angulos alternos externos formados por dos rectas $\ell_1$ y $\ell_2$ y una transversal $t$ son congruentes si y solo si $\ell_1$ y $\ell_2$ son paralelas ($\ell_1 || \ell_2$).

\item Los \'angulos correspondientes formados por dos rectas coplanares y una transversal son congruentes si y solo si las rectas que los forman son paralelas.

\item Los \'angulos consecutivos formados por dos rectas $\ell_1$ y $\ell_2$ y una transversal $t$ son suplementarios si y solo si $\ell_1$ y $\ell_2$ con paralelas ($\ell_1  ||  \ell_2$).
\end{itemize}







\end{document}