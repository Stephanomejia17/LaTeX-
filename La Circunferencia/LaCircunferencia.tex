\documentclass[•]{article}

\begin{document}
	\begin{center}
		\begin{huge}
			La Circunferencia
		\end{huge}
	\end{center}
	
	\begin{flushleft}
		\textbf{Teoremas:}
		\begin{itemize}
			\item Sean $\ell$ y $C(O, r)$ un recta y una circunferencia coplanares, Si $l$ es perpendicular a un segmento radial $\overline{OQ}$ en el punto $Q$ sobre la circunferencia, entonces $\ell$ es una tangente de $C(O, r)$
			\item Toda recta tangente a una circunferencia es perpendicular al segmento radial en el punto de tangencia.
			\item \textbf{Teorema de las tangentes: }Los segmentos tangentes trazados a una circunferencia desde un punto exterior y coplanar con ella, son congruentes y determinan \'angulos congruentes con la recta que contiene el punto y el centro de la circunferencia.
			\item Si $\stackrel{\textstyle\frown}{\mathrm{AB}}$ es una semicircunferencia, entonces m($\stackrel{\textstyle\frown}{\mathrm{AB}}$) = 180
			
			\item En una circuferencia o en circunferencias congruentes, dos \'angulos centrales son congruentes si y s\'olo si los arcos que subtienden son congruentes.
			\item En una circunferencia o en circuferencias congruentes, dos cuerdas son congruentes si y s\'olo si determinan arcos congruentes.
			\item Una recta que pasa por el centro de una circunferencia es perpendicular a una cuerda, si y s\'olo si la biseca
			\item \textbf{Determinaci\'on de la circunferencia: }Por tres puntos no colineales pasa una circunferrencia y solo una.
			\item En una misma circunferencia o en circunferencias congruentes dos cuerdas son congruentes si y solo si equidistan del centro.
			\item La medida de un \'angulo inscrito es igual a la mitad de la medida del arco que subtiende.
			\item La medida de un \'angulo semi-inscrito es igual a la mitad de la medida del arco que subtiende.
			\item La medida de un \'angulo exterior a una circunferencia es igual a la semidiferencia de las medidas de los arcos comprendidos entre sus lados.
			\item La medida de un \'angulo interior a una circunferencia es igual a la semi-suma de la medida del arco subtendido por \'el y la medida del arco subtendido por las prolongaciones de los lados del \'angulo.
			
			\item Dada una circunferencia y un punto exterior da ella, desde donde parten un segmento tangente y uno secante a la circunferencia, la medida del segmento tangente es media proporcional geom\'etrica entre las medidas del segmento secante y su segmento exterio.
			
			\item Dados dos o m\'as segmentos secantes a una circunferencia y que comparten un extremo en un punto exterior a ella, el producto de la medida de cualquiera de los segmentos por la medida de su respectivo segmento exterior, es constante.
			
			\item En una circunferencia, la intersecci\'on de dos cuerdas determina dos segmentos en cada cuerda, de tal manera que el producto de las medidas de los segmentos determinados en una de ellas es igual al producto de las medidas de los segmentos determinados en la otra.
			
			\item En todo cuadril\'atero inscrito en una circunferencia, los \'angulo opuestos son suplementarios.
			
			\item Un cuadril\'atero que tiene dos \'angulos opuestos suplementarios es inscribible.
			
		\end{itemize}
		
		\textbf{Corolarios: }
		\begin{itemize}
			\item Dada una tangente a una circunferencia, toda recta perpendicular a dicha tangente en el punto de tangencia para por el centro del c\'irculo.
			\item Toda recta que pase por el centro de una circunferencia y biseque una cuerda, biseca al arco.
			\item La mediatriz de una cuerda para por el centro de la circunferencia.
			\item Los \'angulos inscritos que subtienden el mismo arco son congruentes.
			\item Todo \'angulo inscrito que subtiende una semi-circunferencia es recto.
			\item Los arcos comprendidos entre dos rectas paralelas que intersecan una circunferencia son congruentes.
		\end{itemize}
		
		
		\textbf{Postulados: }
		\begin{itemize}
			\item \textbf{Adici\'on de Arcos: }Si la intersecci\'on de los arcos  $\stackrel{\textstyle\frown}{\mathrm{AB}}$ y $\stackrel{\textstyle\frown}{\mathrm{BC}}$ de una \textbf{misma} circunferencia es un punto, entonces 
		\end{itemize}
	\end{flushleft}
\end{document}