\documentclass[•]{article}
\begin{document}
\begin{center}
	\begin{huge}
		\textbf{Proporcionalidad y Semejanza}
	\end{huge}
\end{center}

\begin{flushleft}
	\textbf{Teoremas:}
	\begin{itemize}
		\item \textbf{Teorema fundamental de Proporcionalidad: }Si una recta intercepta dos lados de un tri\'angulo en puntos diferentes y es paralela al tercer lado, entonces determina sobre los lados del tri\'angulo segmentos proporcionales.
		\item Una recta que intersecta dos lados de un tri\'angulo determinando segmentos proporcionales en dichos lados es paralela al tercer lado.
		\item \textbf{Teorema de Thales: }Si tres o m\'as paralelas son intersectadas por dos transversales, los segmentos determinados en una de las transversales son proporcionales a los segmentos determinados en la otra.
		\item \textbf{Teorema de la bisectriz(\'Angulo interior): }La bisectriz de un \'angulo interior de un tri\'angulo determina en el lado opuesto segmentos proporcionales a los lados adyacentes.
		\item \textbf{Teorema de la bisectriz(\'Angulo exterior): }La bisectriz de un \'angulo exterior de un tri\'angulo no is\'osceles determina sobre la prolongaci\'on del lado opuesto dos segmentos proporcionales a los lados adyacentes del tri\'angulo.
		\item La semejanza es una relaci\'on de equivalencia.
		\item \textbf{Criterio A-A: }Dos tri\'angulos que tienen dos \'angulos respectivamente congruentes son semejantes.
		\item \textbf{Criterio L-A-L: }Dos tri\'angulos que tienen dos lados correspondientes proporcionales y los \'angulos formados por estos lados congruentes son semejantes.
		\item \textbf{Lado-Lado-Lado (L-L-L): }Si dos tri\'angulos tienen sus lados respectivamente proporcionales son semejantes.
		\item En un tri\'angulo rect\'angulo, la altura relativa a la hipotenusa determina dos tri\'angulos semejantes al tri\'angulo original.
		\item \textbf{Teorema de Pit\'agoras: }En un tri\'angulo rect\'angulo, el \'area del cuadrado construido sobre la hipotenusa es igual a la suma de las \'areas de los cuadrados construidos sobre los catetos.
		\item \textbf{Rec\'iproco del teorema de Pit\'agoras: }Si en un tri\'angulo el \'area del cuadrado construido sobre uno de sus lados es igual a la suma de las \'areas de los cuadrados construidos sobre los otros lados, entonces el tri\'angulo es rect\'angulo.
		\item En cualquier tri\'angulo el producto de la base y la altura correspondiente es independiente de la selecci\'on de la base.
		
		
			 
	\end{itemize}
	\textbf{Corolarios:}
	\begin{itemize}
		\item \textbf{T. Proporcionalidad: }Una recta que intersecta dos lados de un tri\'angulo y es paralela al tercer lado determina un tri\'angulo cuyos lados son proporcionales a los lados del tri\'angulo inicial.
		\item Dos tri\'angulos rect\'angulos que tienen los catetos proporcionales son semejantes.
		\item Dos tri\'angulos is\'osceles que tengan la base y un lado respectivamente proporcionales son semejantes.
		\item Toda recta que intersecta dos lados de un tri\'angulo y es paralela al tercer lado, determina un tri\'angulo semejante al primero.
		\item Los elementos correspondientes de dos tri\'angulos semejantes, (alturas, medianas, bisectrices), son proporcionales a las medidas de dos de los lados correspondientes.
		\item En un tri\'angulo rect\'angulo, la medida de cualquiera de los catetos es media proporcional geom\'etrica entre las medidas de su proyecci\'on sobre la hipotenusa y la hipotunsa.
		\item En un tri\'angulo rect\'angulo, la medida de la altura relativa a la hipotenusa es media proporcional geom\'etrica entre las medidas de las proyecciones de los catetos sobre la hipotenusa.
		
	\end{itemize}

\end{flushleft}


\end{document}